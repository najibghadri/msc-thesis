%--------------------------------------------------------------------------------------
% Feladatkiiras (a tanszeken atveheto, kinyomtatott valtozat)
%--------------------------------------------------------------------------------------
\clearpage
\begin{center}
    \large
    \textbf{FELADATKIÍRÁS}\\
\end{center}


Jelenetértelmezés megvalósítása autonóm járművekhez mély neurális hálók segítségével
A gépi tanulás új módszerei az intelligens érzékelés számos területét forradalmasították az
elmúlt évtizedben. Ezen módszerek közül külön figyelmet érdemel a mély tanulás (Deep
Learning), amely a gépi tanulás legtöbb területén state of the art megoldásnak számít. A mély
tanulás egyik legfontosabb alkalmazása a jelenetértelmezésben (scene understanding)
történik, ahol az környezetben található objektumok minél több jellemzőjét és ezek
kapcsolatait kívánjuk feltárni.

Az utóbbi néhány évben a mély tanulás kutatásának egyik fontos fókusza a különböző mobilis
robotokban történő alkalmazása, amelyek a robot aktuális környezetének átfogó ismeretét
igénylik. Az alkalmazások közül kiemelhető az autonóm járművek területe, ahol az
objektumokról meghatározott információ teljessége és megbízhatósága kiemelten fontos.

A diplomatervezés során a hallgató feladata egy olyan algoritmus késztése, amely képes
bemeneti kép vagy videofolyam alapján az abban található objektumok felismerésére és a
jármű számára releváns információk (sebesség, forma, relációk stb.) kinyerésére.

A hallgató feladatának a következőkre kell kiterjednie:

\begin{itemize}
    \item Tanulmányozza át a téma releváns szakirodalmát. Vizsgálja meg, hogy más műhelyek
          milyen megoldásokat alkalmaznak.
    \item Készítsen rendszertervet egy megoldásra, amely képes a jelenetértelmezés
          elvégzésére.    
    \item Végezze el az algoritmus fejlesztését és tanítását.
    \item Tesztelje a megoldás pontosságát és hatékonyságát, valamint végezze el a tanuló
    algoritmus validációját.    
    \item Vizsgálja a megoldást valósidejűség szempontjából.
\end{itemize}




Tanszéki konzulens: Szemenyei Márton 
