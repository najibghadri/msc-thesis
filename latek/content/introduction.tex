%----------------------------------------------------------------------------
\chapter{\bevezetes}
%----------------------------------------------------------------------------

I am very passionate about artificial intelligence and as much inspired by the
work of tech companies such as Tesla. Tesla has managed create cutting edge
technology, creating compelling and practical electric cars combined with their
Tesla Autopilot system. It has become iconic to sit in a Tesla and watch it
drive itself. Tesla has already driven 3 billion drives on autopilot, their
access to data is most likely number one in the world. There are other important
companies who develop autopilot systems, one of them is MobilEye an Israeli
subsidiary of Intel corporation that was actually a supplier of Tesla until they
set apart in part due to disagreements on how the technology should be built,
which is an important topic that I will talk about.

There are a couple of topics we should establish first. The first being levels
of autopilot systems as defined by SAE (Society of Automotive Engineers)
(\refstruc{fig:J3016}). 

\begin{figure}[!ht]
    \centering
    \includegraphics[width=150mm, keepaspectratio]{figures/levels-of-ad.jpg}
    \caption{Levels of driving automation defined in SAE J3016 \cite{j3016b}}
    \label{fig:J3016}
\end{figure}

From level 0 to 2 are automations where the human is still required to fully
monitor the driving environment. Tesla's autopilot is Level 2 which is partial
automation that includes control of steering and both acceleration and
deceleration. From Level 3 the human is not required to monitor the environment.
Full automation, where the driver is not expected to intervene and the
vehicle is able to handle all situations is on Level 5. In order to achieve that
level the autopilot must fully understand the environment.

This is however very difficult. The algorithms that we know today are not enough
to achieve understanding of the environment yet. Even Convolutional Neural
Networks (CNNs) are not cabale of understanding deep concepts of the world. CNNs
are mainly used in computer vision and are useful when we want to recognize
patterns that appear anywhere in 2D images. Today we are able to calssify
images, detect and localize objects, segment images to very high accuracy,
however this doesn't mean the computer \emph{understands} the scenes.
Furthermore these algorithms are trained very specifically: To build a detection
neural network (NN) first a meticulous dataset must be created that tells the
algorithm what must be detected - we call this the ground truth, or training
data set. Then the NN must be trained and optimized until it yields a low error
on the test dataset. We call this Deep Learning due to the fact that the networks
contain millions of parameters that are trained through hundreds of thousands of
iterations. This is not close to what might be general AI.

In this sense we can argue about the meaning of "scene understanding". There is
research going on in that direction most notably in my opinion by Yann LeCun the
director of Facebook AI and professsor at NYU, who works a concept called
energy-based models. The Energy-based model that is a form of generative model
allow a mathematical "bounding" or "learning" of a data distribution in any
dimension. Upon prediciton the model tries to generate a possible future for the
current model in time, where the generated future model acts as the prediciton
itself. Generative adversarial networks are a type of these models. This is in
contrast to the other main machine learning approach that is the discriminative
model which is what we use mostly. Perceptrons such as NNs and CNNs, support
vector machines fall into this category, however the distinction is still not
clear.

For the purpouse of this thesis hence it is important to define what a system
capabale of understanding scences in driving situations means. The essentials
are the following:
\begin{itemize}
    \item Lane  and path detection
    \item Driveable area detection
    \item Object detection: cars, pedestrians, etc.
    \item Object localization in 3D real world space
    \item Object tracking and identification
    \item Foreign object detection: anything that shouldn't be where it is
    \item Traffic light and sign understanding
    \item Handling occlusion of objects
    \item Pedestrian crossing detection
    \item Knowledge of surroundings and road for example with the help of
          high definition maps
\end{itemize}

In an ideal world, where all cars are autonomous these perceptions would be
enough, however the future of self-driving cars is going to be a transition,
where both humans and machines will drive on the roads. We humans already
account for each other (we try as we can), but self-driving cars will have to
account for us too. We might not be smart but driving on the road sometimes
requires improvization to save a situation and we might need a more general AI.

For the vehicle to understand it's surroundigs first of all it needs sensors.
Each company goes differently about the sensor suite, and it is quite
interesting to examine each solution. I will talk about this in the next
chapter, Related works.

\section{Proposed solution}

As we said to develop our system first of all we need data. There are a lot of
datasets available on the internet for car driving. They include object
detections, segmentations, map data, lidar data. Some of the most notable ones
are the nuScene dataset\footnote{nuScene dataset
    \url{https://www.nuscenes.org/}}, Waymo dataset\footnote{Waymo dataset
    \url{https://waymo.com/open/}} from Google's self-driving car company or the
Cityscapes dataset\footnote{Cityscapes dataset
    \url{https://www.cityscapes-dataset.com/}} and more. Each of these datasets are
very good, however they are not really helpful for our case.

In order to localize object we will need to localize them, and for that I use
stereo imaging. Each AD system today employs stereo camera setting because it is
a very simple a quite accurate way of estimating depth for each pixel in an
image. In order to have the \emph{freedom} to create a custom camera setting I
cannot rely on these datasets. Furthermore, I want to measure the success rate of
my detector however there is no dataset that contains all the necessary
information, because in fact it is not possible to collect everything from the
real world.

This is why I choose to use a \emph{simulation} instead to test the system.
Using a simulation gives me a huge ammount of freedom. My research work started
in looking for simulations that let me extract data from the simulation in each
frame and let's me create custom sensor settings. The first thing that came to
my mind was GTA V the well-known free-world game, but soon I found out it isn't
going to be useful. I found that there are dedicated projects for this, and a
couple of companies have already invested in this. There is Deepdrive from
Voyage auto\footnote{Deepdrive Voyage \url{https://deepdrive.voyage.auto/}}, an
American AD supplier, NVIDIA has a project going on called Drive
Constellation\footnote{NVIDIA Drive Constellation
\url{https://www.nvidia.com/en-us/self-driving-cars/drive-constellation/}} and
another project called RFPro\footnote{RFPro \url{http://www.rfpro.com/}}.
However these are either not opensource or not mature enough. The best that I
found and ended up basing my project on is called CARLA Simulator\footnote{CARLA
Sim \url{http://carla.org/}} \cite{Dosovitskiy17} that is a quite mature and
fully opensource project developed by the Barcelonian university UAB's computer
vision CVC Lab supported by Intel, Toyota, GM and others. This simulator has a
really good API that fulfills my requirements.

\begin{figure}[!ht]
    \centering
    \includegraphics[width=150mm, keepaspectratio]{figures/carla.png}
    \caption{A screenshot from CARLA}
    \label{fig:carla}
\end{figure}

The sensor suite includes 8 RGB cameras mounted on the roof of the car as shown
on \refstruc{fig:3dmodel2}. As the title of the thesis says, I only used RGB
cameras and no other sensors. This is the same approach Tesla is taking,
contrary to almost all other players in the field including MobilEye and Waymo.
Lidar data is good for correction, but it is better if the AI can equally
perform using only RGB cameras since it is a more general solution that is
closer to how we humans percieve the environment.

\begin{figure}[!ht]
    \centering
    \includegraphics[width=150mm, keepaspectratio]{figures/3dmodel2.png}
    \caption{How the cameras are set up on the roof}
    \label{fig:3dmodel2}
\end{figure}

The system uses state-of-the-art detection, localization and segmentation model
Detectron2 \cite{wu2019detectron2} a MASK R-CNN conv net model based on Residual
neural networks and Feature Pyramid Networks trained on the COCO general
dataset\footnote{COCO dataset \url{http://cocodataset.org/}}.

Finally I develop a 3D webvisualizer that lets us replay the ground truth and
detection log simultaneously and compare the error between the two.

\refstruc{fig:flow} depicts this taskflow.

\begin{figure}[!ht]
    \centering
    \includegraphics[width=150mm, keepaspectratio]{figures/flowchart.png}
    \caption{Task flow}
    \label{fig:flow}
\end{figure}

\section{Summary of results}

The result is a detector that is capable of localizing vehicles, and pedestrians
on the road up to 100 meters with an accuracy of ~50cm in an angle of 270\degree
centered to the front. The virtual vehcile has 8 RGB cameras mounted creating 4
stereo sides. The algorithm is written in Python and uses PyTorch, with that on
an NVIDIA Titan X GPU the detector can perform in 2.7FPS for one side, ie. for
two cameras. In an embedded optimized system using C or C++ code this can easily
be improved to even 60FPS creating a real-time system. The code cannot perform
lane detection yet, but that would have been the easier part. The webvisualizer
let's us relplay the simulation frame by frame and see the detection error for
each actor in the scene. It also shows a montage the original, detection and
depthmap. Below \refstruc{fig:webviz1} shows a screenshot of the webvisualizer
in action.

\begin{figure}[!ht]
    \centering
    \includegraphics[width=150mm, keepaspectratio]{figures/webviz2.png}
    \caption{3D wevisualizer}
    \label{fig:webviz1}
\end{figure}

All of the code for the thesis, detector, simulator configuration and
webvisualizer is available on \url{https://github.com/najibghadri/msc-thesis}
and you can access the webvisualizer and interactively replay and test
simulations on \url{https://najibghadri.com/msc-thesis/}.


\section{Thesis structure}

In the next chapter, \autoref{chap:relatedwork} I analyze and
compare two self-driving solutions: Tesla autopilot and MobilEye autopilot.

In \autoref{chap:perceptions} I talk about different kind of perceptions,
state-of-the-art Convolutional Networks and computer vision algorithms that are
useful for our use-case. In \autoref{chap:assumptions} I define the technical
assumptions that I made in order to simplify the task and the resulting
limitations. In \autoref{chap:designimplementation} I detail the design and
implementation of the simulator configuration, the detector algorithm and the
webvisualizer.

Then in \autoref{chap:results} I present different measurements and results, and
in \autoref{chap:experimental} I present experimentations that ended up not
being part of the detection. Finally I discuss ways to improve the system in
\autoref{chap:improvement}

- Structure of Thesis
- Each chapter
- All code and thesis available at https://github.com/najibghadri/msc-thesis and the published verision on my website