%----------------------------------------------------------------------------
\chapter{Conclusion}
\label{chap:conclusion}
%----------------------------------------------------------------------------

The final scene understanding algorithm is not a system that can be applied by
itself in a real scenaro, however it builds on the same basic ideas for scene
understanding for cars. More importantly this thesis help me understand all the
building block required for autonomous driving. The work of companies like Tesla
and Waymo constitues many top researchers in the field. In Hungary this market
is yet in early stages but companies like BOSCH or a smaller company like
AIMotive are already present and working on the field with a good pace.

Working on this thesis has been a unique experience because the whole field was
new to me before diving into it. Usually thesis projects require that the
student works on the same project for 4 semesters, however I had to take a
different path. I did my previous research work in Web Applications and applied
blockchain technology. Then I took an optional a deep learning class and it
sparked my interest for AI even more. Taking this project was a risk and I had
to learn about basic computer vision processing methods, algorithms, 3D vision,
the camera model, convolutional neural networks and deep learning and even a
little bit of game engines because of the simulator. But in the end I learned
valueable things and I hope I can use this knowledge soon in an AI company
perhaps one that works on autopilots.

