\pagenumbering{roman}
\setcounter{page}{1}



\selecthungarian

%----------------------------------------------------------------------------
% Abstract in Hungarian
%----------------------------------------------------------------------------
\chapter*{Kivonat}\addcontentsline{toc}{chapter}{Kivonat}

Az önvezető autók kétség kívűl a jövőt 

\vfill
\selectenglish


%----------------------------------------------------------------------------
% Abstract in English
%----------------------------------------------------------------------------
\chapter*{Abstract}\addcontentsline{toc}{chapter}{Abstract}

Autonomous driving is undoubtedly the future of transportation. The comfort that
it brings us is what drives us to work on making it real. We already have
autonomous systems in public transportation in abundance, but it is different
when we talk about the car roads. Driving a car requires near-human intelligence
due to the nature of the environment, in fact it is impossible to define the
environment. A train's or subways's environment can be defined mathematically
and hence controlled easly, but for a machine to drive a car, it has to
understand what we understand, and what we understand is even hard to define
ourselves.

Computer science has come a long way, and we have already seen the rise of
artificial intelligence algorithms and their effectiveness. Out of these methods
Deep Learning and Convolutional Neural Networks are key tools in achieving our
goal. With these algorithms computers learn general concepts of the world, and
this is essential to make a safe autonomous driving (AD) system. We will see in
this work briefly what they are and how they work. 

Some notable companies have already achieved a high level of AD, most notably
Tesla, and another AD supplier MobilEye. These companies use algorithms that are
developed globally and publicly and I used them in algorithm to partly achieve
what they have achieved. 

In this work I create a Scene Understanding system specialized for driving
situations. I choose to evaluate the system on a virtual car driving simulation
called CARLA Sim, that is going to benefit us to measure our rate of success.

I researched how existing autonomous driving systems have been built, and
inspired by them I designed a system that is capable of recognizing important
information for a car on the road. I used stereo imaging of multiple RGB cameras
mounted on top of our virtual car for depth estimation and used trained
Convolutional Neural Networks to then perform further infomration extraction
from the images and perform detection for each frame of the simulation. I made a
3D webvisualizer that is able to show us the difference between ground truth
information extracted programatically from the simulator and the detection
infomration while simultaneously play a montage video of the simulation. Finally
I evaluated the system and measured it's validity for real situations and
provided further improvement notes on my work. This thesis is also published on
\url{https://najibghadri.com/msc-thesis/} where you can try the 3D
webvisualizer.

\vfill
\selectthesislanguage

\newcounter{romanPage}
\setcounter{romanPage}{\value{page}}
\stepcounter{romanPage}